\section{Аспекты реализации на GPGPU.}
\textbf{Основные способы}
\begin{itemize}
    \item Просто использовать GPU для вычисления приспособленности

    \item Запускать весь алгоритм на GPU
    \begin{itemize}
        \item Использовать больше параллелизма, если функции приспособленности простые
        \item Необходимо минимизировать объем данных, передаваемых между CPU и GPU 
    \end{itemize}
    
\end{itemize}
\textbf{Ключевые особенности GPU }
\begin{itemize}
    \item Топология: не все ядра одинаково легко общаются со всеми другими
    \item Работа с ветвлениями: не весь код одинаково хорошо портируется на GPU
    \begin{itemize}
        \item Например, если мы в основном производим матричные произведения, портирование легко осуществимо
        \item Если мы, например, меняем представление: преобразуем вектор в граф и обратно, такой код будет сложно портировать на GPU
    \end{itemize}
\end{itemize}

Наиболее подходящий алгоритм для реализации на GPU - клеточный алгоритм. 