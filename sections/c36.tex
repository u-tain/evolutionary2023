\section{Проектирование ЭА.}

Все интересные решения должны быть представимы.

Для оператора скрещивания надо понимать, что он делает,
какую роль выполняет.

При выборе оператора мутации стоит обращать внимание на:
\begin{itemize}
    \item Гладкость
    \item Глобальность
    \item Вероятность больших изменений
\end{itemize}

Параметры эволюционных алгоритмов могут быть критичными для
процесса оптимизации.

Если параметров нет - значит, что вы их вписали в код руками.

\subsection*{Инвариант}

\textbf{Алгоритм должен подчиняться инвариантам}, если это
оптимизация черного ящика.

Если нет - тоже должен, но при рассмотрении преобразованной
задачи, в которой учтены знания об исходной задаче.

\begin{itemize}
    \item Сдвиг: если формулировка задачи сдвинута на фиксированный
    вектор, поведение алгоритма не должно измениться.
    \item Отражение, умножение на константу и т.д. -
    \textcolor{red}{Непредвзятость}
    \item Вращение - более сильный инвариант.
\end{itemize}

\subsection*{Не допускай глупостей}

\begin{enumerate}
    \item Наивный пропорциональный отбор
    \begin{itemize}
        \item Вероятность выживание пропорциональна приспособленности
        \item Приспособленность всех особей сходится к узкому интервалу
        (Выбираем из равномерного распределения)
        \item $(\mu, \mu)$-ГА с таким отбором решает OneMax за экспоненту
    \end{itemize}
    \item Заведомо досрочная сходимость
    \begin{itemize}
        \item Вещественнозначная оптимизация
        \item Мутация - гауссова с ограниченной дисперсией
        \item Скрещивание - выпуклая комбинация
        \item Объем выпуклой оболочки популяции уменьшается
        по экспоненте.
        \item Равновесие может наступить не только в локальном
        оптимуме.
    \end{itemize}
\end{enumerate}

\subsection*{Тестирование}

ЭА могут выдавать прилично выглядящие результаты, даже если
они сделаны очень плохо.

Тестировать рандом сложно.

Модульные тесты:
\begin{itemize}
    \item Выделяем детерминированные подзадачи, пишем тесты для них.
    \item Для рандомизированных подзадач выделяем перечень действий
    и проверяем, что все выполняются. (Повтори $n\log n$ раз)
\end{itemize}

Интеграционные тесты:
\begin{itemize}
    \item Решаем известную задачу => Измеряем число запросов
    до нахождения решения.
    \item Учитывая, что величина случайна => берем диапазон с
    запасом.
\end{itemize}