\newcommand\tab[1][1cm]{\hspace*{#1}}
\section{Метод имитации отжига.}
\textbf{Весьма хороший подход для поиска глобальных оптимумов.}\\
Два принципа, позаимствованные из физики металлов \\
1. Иногда применять мутации, ухудшающие текущее решение
- Так можно покидать локальные оптимумы (Позволяет расширить генерируемое потомство и снизить зависимость оптимума от мутации)\\
2. Чем больше времени прошло с начала оптимизации, тем меньше должна быть вероятность ухудшить решение \\
- Начало оптимизации = исследование среды (exploration) \\
- Конец = улучшение достигнутого (exploitation) \\
- Параметр t: «температура», зависит от номера итерации, управляет балансом exploration/exploitation \\

function Next($x,t$)  \\
\tab y ← RandOMMutatIOn(x)  \\ 
\tab if f (y) $\ge$ f (x) or RandOM(0,1) $\le$ $exp((f (y)-f (x))/t)$ then \\ 
\tab \tab return y  \\
\tab end if  \\
\tab return x  \\
end function \\
