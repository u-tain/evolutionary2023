\section{Векторные представления и операторы над ними.}
Векторы могут быть следующие: 
► Булевы
► Вещественные
► Целочисленные
► Смешанные
При выборе вектора стоит задуматься о том какой тип использовать и как их инициализировать и какие операции будем использовать (мутации и тд.)

Булевы это выбор из 0 и 1. 
Возможные операции и рекомендации по ним:
Инициализация
► Чаще всего равновероятный выбор из {0, 1}
► Иногда стоит инициализироваться жадными решениями (пример: рюкзаки)
Операторы мутации:
► Локальная мутация: инвертировать случайно выбранный бит
► Глобальная мутация: инвертировать каждый бит с вероятностью p. Рекомендация по умолчанию: p = \frac{1}{N}
► pi: можно поставить вероятность мутации в зависимость от позиции, аллеля (0 или 1), фидбека от функции приспособленности
► Хорошая идея для незашумленных задач: инвертировать минимум один бит
Для скрещивание стоит использовать следующее: однородное скрещивание с вероятностью обмена p = \frac{1}{2} (то есть равномерно перемешиваем гены 2 родителей)
Вещественные векторы:
В качестве мутации можно использовать функции рандома но надо следить чтобы новые значения были в обалсти определения
Для кроссовера можно воспользоваться следующей формулой: t1 ← axi + (1 − a)y и t2 ← $\beta$yi + (1 − $\beta$)x, где t это гены ребенков а a и $\beta$ задаются в области Rand−p, 1 + p). При этом опять следует следить чтобы новые значения были в области определения
Целочисленные векторы:
То же самое что и с вещесвенными но следует следить чтобы элемент мутации не был равен самому себе
