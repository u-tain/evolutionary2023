\section{Алгоритмы, основанные на использовании индикаторов.}
Виды индикаторов\\
- Унарный индикатор: функция $I$ : $2^{R^n}$ → $R$, сопоставляющая вещественное
значение («качество») множеству (недоминированных) точек.
Принимат популяцию на вход и выдает вещественное число, которое определяет качество опытного набора в сторону максимизации\\
- Бинарный индикатор: функция $I$ : $2^{R^n} ×$ $ 2^{R^n} $ $→R$, сопоставляющая
вещественное значение («различие в качестве») паре множеств
(недоминированных) точек. Это вещественное число можно интерпретировать как различие в качестве между 2-мя популяциями\\

Общая идея эволюционных алгоритмов, основанных на индикаторах\\
- Для унарных индикаторов: принимать такие мутации множества точек x,
чтобы $I(x)$ не убывало. Всегда стремилась к единичке\\
- Для бинарных индикаторов: принимать такие мутации множества точек $a$,
превращающие его в $b$, чтобы $I(a,b) \geq 0$. Чем выше единообразие в поведении индикаторных функций, тем эффективнее работает ЭА и тем больше будет оптимальных решений на полученной популяции.
