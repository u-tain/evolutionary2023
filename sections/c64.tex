\section{Подходы к оптимизации зашумленных функций.}

Основные классы подходов:
\begin{itemize}
    \item Явное усреднение: обсчитываем много раз и усредненяем
    \item Другие виды оценки по выборке
    \item Неявное усреднение: увеличение размеров популяции
    в надежде на повторы
    \item Специализированные стратегии генерации новых особей
    \item Специализированные операторы устойчивые к шуму
\end{itemize}

\subsection*{Явное усреднение}

Размер выборки:
\begin{itemize}
    \item Статическая выборка (одинакова для всех)
    \item Динамическая
    \begin{itemize}
        \item Может зависеть от конкретной особи
        \item Может быть скорректировано
        \item Можно агрегировать критерии
        \item Может зависеть от приспособленности
        \item Может зависеть от других особей в популяции
    \end{itemize}
\end{itemize}


\subsection*{Не усреднение}

Можно использовать другие статистики (медиану).

Можно усреднять по интервалам:
\begin{itemize}
    \item Поделить $[f_{min}; f_{max}]$ на интервалы
    \item Посчитать число попаданий особи в интервал
    \item Взять середины интервалов и сложить с весами
\end{itemize}

Использовать нечеткую логику:
\begin{itemize}
    \item Поделить $[f_{min}; f_{max}]$ на нечеткие множества
    \item Фаззифицировать (ЧЕГО БЛЯТЬ?!) каждое измерение
    \item Усреднить нечеткие результаты и дефаззифицировать результат
\end{itemize}

Использовать оценки приспособленности родителей.


\subsection*{Неявное усреднение}

Кратно увеличиваем размер популяции, используем стандартные
алгоритмы.

Дискретная оптимизация: стандартный оператор мутации порождает
копию особи с константной вероятностью. (Может существовать
множество копий, за счет этого неявно усредняем приспособленность.)

Непрерывная оптимизация: если истинная функция приспособленности
относительно гладкая, близкие особи "помогают"\ друг другу при
усреднении.


\subsection*{Специализированные подходы}

Сделать несколько измерений, построить модель зашумленной функции,
использовать параметры модели.

Проводить статистические тесты, проверяющие значимость гипотезы
о том, что одна особь лучше другой.

И всякая подобная дичь.
