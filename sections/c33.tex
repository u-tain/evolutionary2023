\section{Настройка и управление параметрами ЭА.}

\begin{itemize}
    \item Параметры надо выбирать обоснованно, т.к. плохой выбор может привести к сильному ухудшению времени работы. Параметры подобранные эмпирическим путем априори не будут оптимальными.
    \item Иногда нужно настраивать параметры в процессе работы!
\end{itemize}

\subsection*{Настройка параметров}
Пример постановки задачи
\begin{itemize}
    \item Задачи: линейные псевдобулевы функции на битовых строках длиной n 
    \item Эволюционный алгоритм: (1 + 1)-ЭА, вероятность мутации бита: c/n
\end{itemize}


Известные результаты
\begin{itemize}
    \item c = 1: оптимальное время работы (для данного класса): $(e ± o(1)) · n ln n $
    \item c < 1: плавное ухудшение времени работы: $( \frac{e^c}{c} ± o(1)) · n ln n$
    \item c > 1, $OnEMax$ : плавное ухудшение времени работы (та же формула)
    \item c > 16, некоторые линейные функции: \textbf{экспоненциальное} время работы! 
\end{itemize}


\subsection*{Управление параметрами}
Пример постановки задачи
\begin{itemize}
    \item Задача: $LeADIngonES$: длина префикса из единиц в битовой строке длиной n.
    \item Эволюционный алгоритм: (1 + 1)-ЭА, вероятность мутации бита: $p$
\end{itemize}


Известные результаты
\begin{itemize}
    \item $p = 1/n: \frac{e-1}{2} n^2 ± O(n) \approx 0.86n^2$
    \item $p = (1.59...)/n: \approx 0.77n^2$
    \item Выбор p наилучшим образом в зависимости от приспособленности: \textbf{$\approx 0.68n^2$}
    \item Выбор p \textbf{динамически} в зависимости от факта улучшения: \textbf{$\approx 0.68n^2$}
\end{itemize}
    