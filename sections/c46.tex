\section{k-ограниченные псевдобулевы функции.}
Функция называется \textbf{k-ограниченной} если каждая ее подфункция зависит от не более чем k переменных. Еще одно название:\textbf{ Mk ландшафт}\\
\T{
    Для любой псевдобулевой функции $F(x) = \sum\limits_{i=1}^m f_{i}(x_{i1}, x_{i2},...)$
    существует \textbf{квадратичная} псевдобулева функция
    $G(x,y)=\sum\limits_{i=1}^M \alpha_i[(x|y)_{i1}\land(x|y)_{i2}]$ такая что,
    если $(x^*|y^*)$ точка максимума $G$ то $x^*$ точка максимума $F$
}

Это значит, что мы можем выявить значения комбинаций парных переменных и эта функция будет отвечать следующим требованиям: если у нас есть некоторая точка максимума, то и исходная функция будет иметь максимум в этой точке.

\textbf{k-ограниченность}: каждая подфункция зависит не более чем от k переменных.

Любая псевдобулева функция может быть преобразована в квадратичную с
«такими же» оптимумами:\\
т.е. мы можем использовать прямое и обратное преобразование функции и проведение поиска оптимума на той функции, на которой это возможно.
k-ограниченность м.б. \textbf{абсолютной}, когда у нас все переменные независимы, так и \textbf{относительной}. Относительность обычно проявляется на большом наборе аргументов.
\begin{itemize}
    \item[+] $k=2$ (явно ограничиваем набор аргументов)
    \item[+] можно использовать один и тот же оптимизатор для всего
    \item[-] если подфункции зависят от большого числа переменных, можно потратить кучу времени на вычисление таблиц истинности. 
    \item[-] преобразование тоже может занять время
    \item[-] коэффициенты могут стать большими (по размерности), будет зашумлять поиск максимума
    \item[-] минимизация числа вспомогательных переменных NP трудная задача ( когда k-ограниченность относительная, есть некоторые переменные не накладывают зависимость на функции - их нужно сминимизировать.) Почему это трудно? - для каждой такой переменной надо доказать что она не влияет на дальнейший оптимизационный процесс. Это редкий случай. Обычно k-ограниченность является абсолютной.
\end{itemize}
\textit{Ну и так еще одни ограничения, но это уже так, отдельно взятый случай}\\
Похоже на преобразование SAT задачи в 3SAT задачи\\
Любая 3SAT задача может быть сведена к оптимизации квадратичной псевдобулевой функции значит, даже случай $k=2$ NP труден \\
На практике, задачи с константным k возникают достаточно часто, имеет смысл
их изучать (быть может, даже без сведения к $k=2$)
